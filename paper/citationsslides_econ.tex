\documentclass{beamer}

\mode<presentation>
{
  \usetheme{Warsaw}
  % or ...

  \setbeamercovered{transparent}
  % or whatever (possibly just delete it)
}

\usepackage{standalone}
\usepackage{tikz}
\usepackage[english]{babel}
% or whatever

\usepackage[utf8]{inputenc}
% or whatever
\usepackage[outdir=./]{epstopdf}
\usepackage{times}
\usepackage[T1]{fontenc}
% Or whatever. Note that the encoding and the font should match. If T1
% does not look nice, try deleting the line with the fontenc.


\title[Data Sharing and Citations] % (optional, use only with long paper titles)
{Data Sharing and Citations}

\subtitle
{Causal Evidence} % (optional)

\author[Christensen, Dafoe, Miguel] % (optional, use only with lots of authors)
{Garret Christensen\inst{1} \and Allan Dafoe\inst{2} \and Edward Miguel\inst{3}}
% - Use the \inst{?} command only if the authors have different
%   affiliation.

\institute[Universities of Somewhere and Elsewhere] % (optional, but mostly needed)
{
  \inst{1}%
  Berkeley Institute for Data Science, UC Berkeley
  \and
  \inst{2}%
  Department of Political Science, Yale University
  \and
  \inst{3}%
  Department of Economics, UC Berkeley}
% - Use the \inst command only if there are several affiliations.
% - Keep it simple, no one is interested in your street address.

\date[Short Occasion] % (optional)
{December 2017 BITSS Annual Meeting}

\subject{Talks}
% This is only inserted into the PDF information catalog. Can be left
% out. 



% If you have a file called "university-logo-filename.xxx", where xxx
% is a graphic format that can be processed by latex or pdflatex,
% resp., then you can add a logo as follows:

% \pgfdeclareimage[height=0.5cm]{university-logo}{university-logo-filename}
% \logo{\pgfuseimage{university-logo}}



% Delete this, if you do not want the table of contents to pop up at
% the beginning of each subsection:
%\AtBeginSubsection[]
%{
%  \begin{frame}<beamer>{Outline}
%    \tableofcontents[currentsection,currentsubsection]
%  \end{frame}
%}


% If you wish to uncover everything in a step-wise fashion, uncomment
% the following command: 

%\beamerdefaultoverlayspecification{<+->}


\begin{document}
%{ % all template changes are local to this group.
%    \setbeamertemplate{navigation symbols}{}
%    \begin{frame}[plain]
%        \begin{tikzpicture}[remember picture,overlay]
%            \node[at=(current page.center)] {
%                \href{https://www.bitss.org/}{\includegraphics[width=\paperwidth]{../images/BITSSlogo.png}}
%            };
%        \end{tikzpicture}
%     \end{frame}
%}
%
%\begin{frame}
%  \titlepage
%  \begin{center}
%  \begin{large}
%  PRELIMINARY--Please do not cite.
%  \end{large}
%  \end{center}
%\end{frame}
%
%\begin{frame}
%Thanks to David Birke, Mu Yang Shin, Don Sun, Manana Hakobyan, Terri Cruz, Maxim Guzman, Baiyue Cao, Evey Huang, Rachel Kim, Ravina Pattni, Kevin Khuu
%\end{frame}
%
%%\begin{frame}{Outline}
% % \tableofcontents
%  %\begin{center}
% % Slides available at: \url{http://www.github.com/bitss/citations}
%  %\begin{large}
% % PRELIMINARY--Please do not cite.
%  %\end{large}
% % \end{center}
%  % You might wish to add the option [pausesections]
%%\end{frame}
%
%
%% Since this a solution template for a generic talk, very little can
%% be said about how it should be structured. However, the talk length
%% of between 15min and 45min and the theme suggest that you stick to
%% the following rules:  
%
%% - Exactly two or three sections (other than the summary).
%% - At *most* three subsections per section.
%% - Talk about 30s to 2min per frame. So there should be between about
%%   15 and 30 frames, all told.
%


{ % all template changes are local to this group.
    \setbeamertemplate{navigation symbols}{}
    \begin{frame}[plain]
        \begin{tikzpicture}[remember picture,overlay]
            \node[at=(current page.center)] {
                \includegraphics[width=\paperwidth]{../output/econ_cite_histo.eps}

            };
        \end{tikzpicture}
     \end{frame}
     
     \begin{frame}[plain]
        \begin{tikzpicture}[remember picture,overlay]
            \node[at=(current page.center)] {
                \includegraphics[width=\paperwidth]{../output/econ_cite_time.eps}

            };
        \end{tikzpicture}
     \end{frame}
    
    \begin{frame}[plain]
        \begin{tikzpicture}[remember picture,overlay]
            \node[at=(current page.center)] {
                \includegraphics[width=\paperwidth]{../output/econ_avail_time_dataarticle.eps}
            };
        \end{tikzpicture}
     \end{frame}
}




\section{Results}
\begin{frame}
	\begin{itemize}
		\item Naive OLS results
		\item First stage
		\item 2SLS
		\item Exclusion Restriction
	\end{itemize}
\end{frame}

%\begin{frame}{}
%	\begin{center}
%		\scalebox{0.7}{\input{../output/econ_naive.tex}}
%	\end{center}
%\end{frame}

\begin{frame}{Naive OLS}
	\begin{center}
		\scalebox{0.75}{\input{../output/econ_naive-simp.tex}}
	\end{center}
\end{frame}

%\begin{frame}{}
%\scalebox{0.75}{\input{../output/econ_naiveLN.tex}}
%\end{frame}

\begin{frame}{Naive OLS LN}
	\begin{center}
		\scalebox{0.8}{\input{../output/econ_naiveLN-simp.tex}}
	\end{center}
\end{frame}

\begin{frame}{First Stage}
	\begin{center}
		\scalebox{0.90}{\input{../output/econ_first.tex}}
	\end{center}
	
	\begin{itemize}
		\item  		Controls cut for space
	\end{itemize}
\end{frame}

\begin{frame}{2SLS}
	\begin{center}
		\scalebox{0.75}{\input{../output/econ_ivreg-simp.tex}}
	\end{center}
\end{frame}

\begin{frame}{2SLS LN}
	\begin{center}
		\scalebox{0.7}{\input{../output/econ_ivregLN-simp.tex}}
	\end{center}
\end{frame}

%\begin{frame}{Citation Data}
%Citations are interesting data in and of themselves.
%\begin{itemize}
%\item Multiple sources
%\item Not open
%\item Initiative to change that: \href{http://i4oc.org}{I4OC}
%\end{itemize}
%Results robust to source of citation data
%\end{frame}

{ % all template changes are local to this group.
    \setbeamertemplate{navigation symbols}{}
    \begin{frame}[plain]
        \begin{tikzpicture}[remember picture,overlay]
            \node[at=(current page.center)] {
                \includegraphics[width=\paperwidth]{../output/econ_topicXjournalXpost2005.eps}
            };
        \end{tikzpicture}
     \end{frame}
}

{ % all template changes are local to this group.
    \setbeamertemplate{navigation symbols}{}
    \begin{frame}[plain]
        \begin{tikzpicture}[remember picture,overlay]
            \node[at=(current page.center)] {
                \includegraphics[width=\paperwidth]{../output/econ_typeXjournalXpost2005.eps}
            };
        \end{tikzpicture}
     \end{frame}
}

{ % all template changes are local to this group.
    \setbeamertemplate{navigation symbols}{}
    \begin{frame}[plain]
        \begin{tikzpicture}[remember picture,overlay]
            \node[at=(current page.center)] {
                \includegraphics[width=\paperwidth]{../output/econ_rankXjournalXpost2005.eps}
            };
        \end{tikzpicture}
     \end{frame}
}


\begin{frame}{Exclusion Restriction}
	\scalebox{0.61}{\input{../output/econ_exclusion.tex}}
	\begin{itemize}
		\item Controls dropped for space
	\end{itemize}
\end{frame}

\end{document}


